\subsection{\texttt{dsl} notation}

The well-typed interpreter in Section~\ref{sect:interp} is a simple example of a common programming pattern with dependent types.
Namely: describe an \emph{object language} and its type system with dependent types to guarantee that only well-typed programs can be represented, then program using that representation.
Using this approach we can, for example, write programs for serialising binary data~\cite{plpv11} or running concurrent processes safely~\cite{cbconc-fi}.

Unfortunately, the form of object language programs makes it rather hard to program this way in practice.
Recall the factorial program in \texttt{Expr} for example:

\begin{code}
fact : Expr G (TyFun TyInt TyInt)
fact = Lam (If (Op (==) (Var Stop) (Val 0))
               (Val 1) (Op (*) (app fact (Op (-) (Var Stop) (Val 1)))
                               (Var Stop)))
\end{code}

\noindent
Since this is a particularly useful pattern, \Idris{} provides syntax overloading~\cite{res-dsl-padl12} to make it easier to program in such object languages:

\begin{code}
mkLam : TTName -> Expr (t::g) t' -> Expr g (TyFun t t')
mkLam _ body = Lam body

dsl expr
    variable    = Var
    index_first = Stop
    index_next  = Pop
    lambda      = mkLam
\end{code}

\noindent
A \texttt{dsl} block describes how each syntactic construct is represented in an object language.
Here, in the \texttt{expr} language, any variable is translated to the \texttt{Var} constructor, using \texttt{Pop} and \texttt{Stop} to construct the de Bruijn index (i.e., to count how many bindings since the variable itself was bound); and any \Idris{} lambda is translated to a \texttt{Lam} constructor.
The \texttt{mkLam} function simply ignores its first argument, which is the name that the user chose for the variable.
It is also possible to overload \texttt{let} and dependent function syntax (\texttt{pi}) in this way.
We can now write \texttt{fact} as follows:

\begin{code}
fact : Expr G (TyFun TyInt TyInt)
fact = expr (\x => If (Op (==) x (Val 0))
                      (Val 1) (Op (*) (app fact (Op (-) x (Val 1))) x))
\end{code}

\noindent
In this new version, \texttt{expr} declares that the next expression will be overloaded.
We can take this further, using idiom brackets, by declaring:

\begin{code}
(<$>) : (f : Lazy (Expr G (TyFun a t))) -> Expr G a -> Expr G t
(<$>) f a = App f a

pure : Expr G a -> Expr G a
pure = id
\end{code}

\noindent
Note that there is no need for these to be part of an instance of \texttt{Applicative}, since idiom bracket notation translates directly to the names \texttt{<*>} and \texttt{pure}, and ad-hoc type-directed overloading is allowed.
We can now say:

\begin{code}
fact : Expr G (TyFun TyInt TyInt)
fact = expr (\x => If (Op (==) x (Val 0))
                      (Val 1) (Op (*) [| fact (Op (-) x (Val 1)) |] x))
\end{code}

\noindent
With some more ad-hoc overloading and type class instances, and a new syntax rule, we can even go as far as:

\begin{code}
syntax "IF" [x] "THEN" [t] "ELSE" [e] = If x t e

fact : Expr G (TyFun TyInt TyInt)
fact = expr (\x => IF x == 0 THEN 1 ELSE [| fact (x - 1) |] * x)
\end{code}
